\documentclass{article}
\usepackage{hwopt}
\usepackage{amsmath}	% Package for AMS
\usepackage{amsthm}     % Package for AMS-therom
\usepackage{amssymb}	% Package for AMS-symbol
\usepackage{bm}

\newcommand{\xB}{\bm{x}}
\newcommand{\yB}{\bm{y}}
\newcommand{\RBB}{\mathbb{R}}
\newcommand{\FM}{\mathcal{F}}
\newcommand{\functiontype}[3]{\FM_{#1}^{#2,#3}(\RBB^n)}
\newcommand{\normgen}[1]{\left\| #1 \right\|}

%%%%%%%%%%%%%%%%%
%     Title     %
%%%%%%%%%%%%%%%%%
\title{Mid-term Exam for \emph{Introductory Lectures on Optimization}}
\author{Xiaoyu Wang \\ 3220104364}
\date{\today}

\begin{document}
\maketitle

\begin{excercise}\label{e0}
Proof that if $f_i(\xB)$, $i \in I$, are convex, then
\[
g(\xB) = \max_{i \in I} f_i(\xB)
\]
is also convex.
\end{excercise}
\begin{PROOF}{e0}The answer is as follows:

	$\because$ $\forall i \in I$, $f_i(\xB)$ is convex
	\[
		\therefore \forall i\in I,\ \alpha \in [0,1],\xB ,\yB \in \text{dom} f_i ,f_i(\alpha\xB+(1-\alpha)\yB)\le \alpha f_i(\xB)+(1-\alpha)f_i(\yB)
	\]
	\[
		\because g(\xB) = \max_{i \in I} f_i(\xB)
	\]
	\[
		f_i(\alpha\xB+(1-\alpha)\yB)\le \alpha f_i(\xB)+(1-\alpha)f_i(\yB)\le \alpha g(\xB)+(1-\alpha)g(\yB)
	\]
	So, we have $\forall i\in I,\ \alpha \in [0,1] $ 
	\[
		g(\alpha \xB+(1-\alpha)\yB)= \max_{i\in I} f_i(\alpha\xB+(1-\alpha)\yB)  \le \alpha g(\xB)+(1-\alpha)g(\yB)
	\]
	So $g(\xB)$ is also convex.

\end{PROOF}

\begin{excercise}\label{e1}
Proof that 
\begin{enumerate}
%
\item  if $f$ is a convex function on $\RBB^n$ and  $F(\cdot)$ is a convex and non-decreasing function on $\RBB$, then $g(\xB) = F(f(\xB))$ is convex.
%
\item If $f_i, i=1,\ldots, m$ are convex functions on $\RBB^n$ and  $F(\yB_1, \ldots, \yB_m)$ is convex and non-decreasing (component-wise) in each argument, then 
\[
g(\xB) = F(f_1(\xB), \ldots, f_m(\xB))
\]
is convex.
\end{enumerate}
\end{excercise}
\begin{PROOF}{e1}The answer is as follows:

	\begin{enumerate}
		\item Proof of 1\\
		​    $\because f(\xB)$ is convex and $F$ is non-decreasing
		​    \[
		​      \therefore \forall \alpha \in [0,1]\ ,\xB ,\yB \in \text{dom} f ,f(\alpha\xB+(1-\alpha)\yB)\le \alpha f(\xB)+(1-\alpha)f(\yB)
		​    \]
		​    \[
		​      \therefore g(\alpha \xB+(1-\alpha) \yB)=F\left( (f\left(\alpha \xB+(1-\alpha) \right)\yB)\right)\le F\left(\alpha f(\xB)+(1-\alpha)(f(\yB))\right)\\ 
		​    \]
		​    $\because F(\cdot)$ is convex
		​    \begin{align}
		​      \therefore g(\alpha \xB+(1-\alpha) \yB) &\le F\left(\alpha f(\xB)+(1-\alpha)(f(\yB))\right) \\
		​      &\le \alpha F\left(f(\xB)\right)+(1-\alpha)F\left(f(\yB)\right)\\
		​      &=\alpha g(\xB)+(1-\alpha)g(\yB)
		​    \end{align}
		​    \[ 
		​      \therefore \forall \alpha \in [0,1],\xB ,\yB \in \text{dom} f,\ 
		​      g(\alpha \xB+(1-\alpha) \yB)\le \alpha g(\xB)+(1-\alpha)g(\yB)
		​    \]
		​    \[
				\therefore g(\xB) \text{is convex.}
			\]
		\item Proof of 2\\
		​    $\because f(\xB)$ is convex and $F$ is non-decreasing
		​    \[
		​      \therefore \forall \alpha \in [0,1] ,\xB ,\yB \in \text{dom} f ,f(\alpha\xB+(1-\alpha)\yB)\le \alpha f(\xB)+(1-\alpha)f(\yB)
		​    \]
		​    \begin{align}
				\therefore g(\alpha \xB+(1-\alpha) \yB)&=F(f_1(\alpha \xB+(1-\alpha) \yB),\cdots,f_m(\alpha \xB+(1-\alpha) \yB))\\
						&\le F(\alpha f_1(\xB)+(1-\alpha)(f_1(\yB)),\cdots,\alpha f_m(\xB)+(1-\alpha)(f_m(\yB)))
			\end{align}
		​    $\because F(\xB_1, \ldots, \xB_m)$ is convex
		​    \begin{align}
		​      \therefore g(\alpha \xB+(1-\alpha) \yB)&\le F(\alpha f_1(\xB)+(1-\alpha)(f_1(\yB)),\cdots,\alpha f_m(\xB)+(1-\alpha)(f_m(\yB)))\\
​      			&\le \alpha F(f_1(\xB),\cdots,f_m(\xB))+(1-\alpha)F(f_1(\yB),\cdots,f_m(\yB))\\
				&=\alpha g(\xB)+(1-\alpha)g(\yB)
		​    \end{align}
		​    \[ 
		​      \therefore \forall \alpha \in [0,1],\xB ,\yB \in \text{dom} f,\ 
		​      g(\alpha \xB+(1-\alpha) \yB)\le \alpha g(\xB)+(1-\alpha)g(\yB)
		​    \]
		​    \[
				\therefore g(\xB) \text{is convex.}
			\]
	\end{enumerate}

\end{PROOF}

\begin{excercise}\label{e2}
Proof that if $f(\xB, \yB)$ is convex in $(\xB, \yB) \in \RBB^n$ and  $Y$ is a convex set, then 
\[
g(\xB) = \inf_{\yB \in Y}f(\xB, \yB)
\]
is convex.
\end{excercise}
\begin{PROOF}{e2}The answer is as follows:

	For $\xB \in \RBB^n$,there exists a sequence $\left\{\yB_n\right\}$ such that:
	\[
		\inf_{\yB \in Y}f(\xB,\yB)=\lim_{n\rightarrow \infty}f(\xB,\yB_n)	
	\]
	So we assume $\bm{x_1},\bm{x_2} \in \RBB^n $,  $\bm{y}^{\bm{(x_1)}}_n,\bm{y}^{\bm{(x_2)}}_n$ :
	\[
		\inf_{\bm{y}\in Y}f(\bm{x_1},\bm{y})=\lim_{n\rightarrow \infty}f(\bm{x_1},\bm{y}_n^{\bm{(x_1)}})
	\]
	\[
		\inf_{\bm{y}\in Y}f(\bm{x_2},\bm{y})=\lim_{n\rightarrow \infty}f(\bm{x_2},\bm{y}_n^{\bm{(x_2)}})
	\]
	now we $\forall \alpha \in [0,1] $, we have:
	\begin{align}
		\alpha g(\bm{x_1})+(1-\alpha)g(\bm{x_2})&=\alpha \inf_{\bm{y}\in Y}f(\bm{x_1},\bm{y})+(1-\alpha)\inf_{\bm{y}\in Y}f(\bm{x_2},\bm{y})\\
		&= \alpha \lim_{n\rightarrow \infty}f(\bm{x_1},\bm{y_n}^{\bm{(x_1)}})+(1-\alpha)\lim_{n\rightarrow \infty}f(\bm{x_2},\bm{y_n}^{\bm{(x_2)}}) \\
		&= \lim_{n\rightarrow \infty}(\alpha f(\bm{x_1},\bm{y_n}^{\bm{(x_1)}})+(1-\alpha)f(\bm{x_2},\bm{y_n}^{\bm{(x_2)}}) )
	\end{align}
	Since of $f(\bm{x},\bm{y})$ is convex, we have:
	\begin{align}
		\alpha g(\bm{x_1})+(1-\alpha)g(\bm{x_2})&=\lim_{n\rightarrow \infty}(\alpha f(\bm{x_1},\bm{y_n}^{\bm{(x_1)}})+(1-\alpha)f(\bm{x_2},\bm{y_n}^{\bm{(x_2)}}) )\\
		&\ge \lim_{n\rightarrow \infty}f(\alpha \bm{x_1}+(1-\alpha)\bm{x_2},\alpha \bm{y_n}^{\bm{(x_1)}}+(1-\alpha)\bm{y_n}^{\bm{(x_2)}})
	\end{align}
	$\because Y$ is a convex set, we have:
	\begin{align}
		\alpha g(\bm{x_1})+(1-\alpha)g(\bm{x_2})&\ge \lim_{n\rightarrow \infty}f(\alpha \bm{x_1}+(1-\alpha)\bm{x_2},\alpha \bm{y_n}^{\bm{(x_1)}}+(1-\alpha)\bm{y_n}^{\bm{(x_2)}})\\
		&\ge \inf_{\bm{y}\in Y}f(\alpha \bm{x_1}+(1-\alpha)\bm{x_2},\bm{y})\\
		&=g(\alpha \bm{x_1}+(1-\alpha) \bm{x_2})
	\end{align}
	​    \[ 
	​      \therefore \forall \alpha \in [0,1],\bm{x_1},\bm{x_2} \in \RBB^n,\ 
	​      \alpha g(\bm{x_1})+(1-\alpha)g(\bm{x_2}) \ge g(\alpha \bm{x_1}+(1-\alpha) \bm{x_2})
	​    \]
	​    \[
			\therefore g(\xB) \text{is convex.}
		\]

\end{PROOF}

\begin{excercise}\label{e3}
	Proof that the following univariate functions are in the set of $\mathcal{F}^1(\mathbb{R})$:
	\begin{align}
		f(x) &= e^x,\nonumber \\
		f(x) &= |x|^p,\; p > 1,\nonumber \\
		f(x) &= \frac{x^2}{1 + |x|},\nonumber \\
		f(x) &= |x| - \ln (1 + |x|).\nonumber
	\end{align}
\end{excercise}
\begin{PROOF}{e3}The answer is as follows:
	
	if univariate functions $f$ are in the set of $\mathcal{F}^1(\mathbb{R})$ if and only if:
	
	$\forall x,y \in \mathbb{R}$ 

	\begin{equation}
		\langle\nabla f(x)-\nabla f(y),x-x\rangle\ge 0
	\end{equation}

	That is:
	\begin{equation}
		\left( f^{'}(x)-f^{'}(y)\right) (x-y)\ge 0 
	\end{equation}

\begin{enumerate}
	\item $f(x)=e^x$
	
		\[
			f^{'}(x)=e^x
		\]
		Because $e^x$ is a increasing function, so we have:
		\begin{equation}
			\left( f^{'}(x)-f^{'}(y)\right) (x-y) =(e^x-e^y)(x-y)\ge 0
		\end{equation}
		So $f(x)=e^x$ is in the set of $\mathcal{F}^1(\mathbb{R})$
	\item $f(x)=\lvert x\rvert^p,p>1$
	
		\begin{equation}
			f{'}(x)=\begin{cases}
				p (x^{p-1})&x>0\\
				0&x=0\\
				-p\vert x\vert^{p-1}&x<0
			\end{cases}
		\end{equation}
		we can merge the above equation into:

		\begin{equation}
			f{'}(x)=\begin{cases}
				p(x^{p-1})&x\ge 0\\
				-p\vert x\vert^{p-1}&x<0
			\end{cases}
		\end{equation}

		Because $p>1$, so $x^{p}$,$x^{p-1}$ are a increasing function, so we have:
		\begin{enumerate}
			\item when $x\ge 0,y\ge 0$:
			
			$\left( f^{'}(x)-f^{'}(y)\right) (x-y) = p(x^{p-1}-y^{p-1})(x-y)\ge 0$
			\item when $x\ge 0,y<0$
			
			$\left( f^{'}(x)-f^{'}(y)\right) (x-y) = p(x^{p-1}+\vert y\vert ^{p-1})(x-y)\ge 0$
			\item when $x<0,y\ge 0$
			
			$\left( f^{'}(x)-f^{'}(y)\right) (x-y) = p(-\vert x\vert ^{p-1}-y^{p-1})(x-y) \ge 0$

			\item when $x<0,y<0$
			
			$\left( f^{'}(x)-f^{'}(y)\right) (x-y) = p(-\vert x\vert ^{p-1}+\vert y\vert ^{p-1})(x-y)\ge 0$

		\end{enumerate}
		So we have 
		\[\left( f^{'}(x)-f^{'}(y)\right) (x-y)\ge 0\]
		
		So $f(x)=\lvert x\rvert^p,p>1$ is in the set of $\mathcal{F}^1(\mathbb{R})$
	\item $f(x)=\frac{x^2}{1+\vert x\vert}$
		
		\begin{equation}
			f^{'}(x)=\begin{cases}
				1-\frac{1}{(1+\vert x\vert)^2}&x>0\\
				0&x=0\\
				\frac{1}{(1+\vert x\vert)^2}-1&x<0
			\end{cases}
		\end{equation}
		we can merge the above equation into:
		\begin{equation}
			f^{'}(x)=\begin{cases}
				1-\frac{1}{(1+\vert x\vert)^2}&x\ge 0\\
				\frac{1}{(1+\vert x\vert)^2}-1&x<0
			\end{cases}
		\end{equation}
		\begin{enumerate}
			\item when $x\ge 0,y\ge 0$
			
			$\left( f^{'}(x)-f^{'}(y)\right) (x-y)=(\frac{1}{(1+\vert x\vert)^2}-\frac{1}{(1+\vert y\vert)^2})(x-y)\ge 0 $
			\item when $x\ge 0,y<0$
			
			$\left( f^{'}(x)-f^{'}(y)\right) (x-y)=(2-\frac{1}{(1+\vert x\vert)^2}-\frac{1}{(1+\vert y\vert)^2})(x-y)$
			
			since of $x>y$ and $\frac{1}{(1+\vert x\vert)^2}\le 1$,so we get:
			
			$\left( f^{'}(x)-f^{'}(y)\right) (x-y)=(2-\frac{1}{(1+\vert x\vert)^2}-\frac{1}{(1+\vert y\vert)^2})(x-y)\ge 0$
			
			\item when $y\ge 0,x<0$ 
			
			$\left( f^{'}(x)-f^{'}(y)\right) (x-y)=(\frac{1}{(1+\vert x\vert)^2}+\frac{1}{(1+\vert y\vert)^2} - 2)(x-y)$
			
			since of $y>x$ and $\frac{1}{(1+\vert x\vert)^2}\le 1$,so we get:
			
			$\left( f^{'}(x)-f^{'}(y)\right) (x-y)=(\frac{1}{(1+\vert x\vert)^2}+\frac{1}{(1+\vert y\vert)^2} - 2)(x-y)\ge 0$

			\item when $x<0,y<0$
			
			$\left( f^{'}(x)-f^{'}(y)\right) (x-y)=(\frac{1}{(1+\vert y\vert)^2}-\frac{1}{(1+\vert x\vert)^2})(x-y)\ge 0$
		\end{enumerate}

		So we have 
		\[\left( f^{'}(x)-f^{'}(y)\right) (x-y)\ge 0\]
		
		So $f(x)=\frac{x^2}{1+\vert x\vert}$ is in the set of $\mathcal{F}^1(\mathbb{R})$

	\item $f(x)=\vert x\vert -ln(1+\vert x\vert)$
	
	\begin{equation}
		f^{'}(x)=\begin{cases}
			1-\frac{1}{1+\vert x\vert}&x>0\\
			0&x=0\\
			\frac{1}{1+\vert x\vert}-1&x<0
		\end{cases}
	\end{equation}
	we can merge the above equation into:
	\begin{equation}
		f^{'}(x)=\begin{cases}
			1-\frac{1}{1+\vert x\vert}&x\ge 0\\
			\frac{1}{1+\vert x\vert}-1&x<0
		\end{cases}
	\end{equation}
	\begin{enumerate}
		\item when $x\ge 0,y\ge 0$
		
		$\left( f^{'}(x)-f^{'}(y)\right) (x-y)=(\frac{1}{1+\vert x\vert}-\frac{1}{1+\vert y\vert})(x-y)\ge 0$
		\item when $x\ge 0,y<0$
		
		$\left( f^{'}(x)-f^{'}(y)\right) (x-y)=(2-\frac{1}{1+\vert y\vert}-\frac{1}{1+\vert x\vert})(x-y)$
		
		since of $x>y$ and $f(x)=\frac{1}{1+\vert x\vert}\le 1$,so we get:

		$\left( f^{'}(x)-f^{'}(y)\right) (x-y)=(2-\frac{1}{1+\vert y\vert}-\frac{1}{1+\vert x\vert})(x-y)\ge 0$

		\item when $x<0,y\ge 0$:
		
		$\left( f^{'}(x)-f^{'}(y)\right) (x-y)=(\frac{1}{1+\vert x\vert}-\frac{1}{1+\vert y\vert})(x-y)\ge 0$

		since of $y>x$ and $f(x)=\frac{1}{1+\vert x\vert}\le 1$,so we get:

		$\left( f^{'}(x)-f^{'}(y)\right) (x-y)=(\frac{1}{1+\vert x\vert}-\frac{1}{1+\vert y\vert})(x-y)\ge 0$
		
		\item when $x<0,y<0$:
		
		$\left( f^{'}(x)-f^{'}(y)\right) (x-y)=(\frac{1}{1+\vert y\vert}-\frac{1}{1+\vert x\vert})(x-y)\ge 0$
	\end{enumerate}
	So we have 
	\[\left( f^{'}(x)-f^{'}(y)\right) (x-y)\ge 0\]
	
	So $f(x)=\vert x\vert -ln(1+\vert x\vert)$ is in the set of $\mathcal{F}^1(\mathbb{R})$

\end{enumerate}





\end{PROOF}

\begin{excercise}\label{e4}
For $f \in \functiontype{L}{1}{1}$ and function $\phi(\yB) = f(\yB) - \innerproduct{\nabla f(\xB_0)}{\yB}$, prove that $\phi \in \functiontype{L}{1}{1}$, and its optimal point is $\yB^* = \xB_0$.
\end{excercise}
\begin{PROOF}{e4}The answer is as follows:

	To prove that $\phi \in \functiontype{L}{1}{1}$, we need to prove that $\phi(\yB)$ is convex and $\phi (y)\in C_{L}^{1,1}(\RBB^n)$
	
	According to the definition of $\phi(\yB) = f(\yB) - \innerproduct{\nabla f(\xB_0)}{\yB}$, we have:
	
	\begin{equation}
		\nabla \phi(\yB)=\nabla f(\yB)-\nabla f(\bm{x_0})
	\end{equation}

	\begin{enumerate}
		\item  $\phi (y)$ is convex.
		
		$f\in \mathcal{F}_{L}^{1,1}(\mathbb{R}^n)$, since of its convexity, we have:
		\[
			\langle\nabla f(\xB)-\nabla f(\yB),\xB-\yB\rangle\ge 0
		\]
		So we can get:
		\begin{equation}
			\langle\nabla \phi(\xB)-\nabla \phi(\yB),\xB-\yB\rangle=\langle\nabla f(\xB)-\nabla f(\yB),\xB-\yB\rangle\ge 0
		\end{equation}
		So $\phi (x)$ is convex.
		\item $\phi (y)\in C_{L}^{1,1}(\RBB^n)$
		
		firstly, $\phi (y)$ is continuously differentiable:
		\[ \because f\in \mathcal{F}_{L}^{1,1}(\mathbb{R}^n)	\]
		\[ \therefore f(\xB) \text{ is continuously differentiable}\]
		\[ \therefore \nabla f(\xB) \text{exists}\]
		\[ \therefore \nabla \phi(\yB)=\nabla f(\yB)-\nabla f(\bm{x_0}) \text{ is continuously differentiable}\]
		$f\in \mathcal{F}_{L}^{1,1}(\mathbb{R}^n)$, since of its Lipschitz continuity, we have:

		secondly, $\phi(y)$ satisfies the Lipschitz continuous with constant $L$.
		\[ \because f\in \mathcal{F}_{L}^{1,1}(\mathbb{R}^n)	\]
		\[ \therefore \vert \nabla f(\xB)-\nabla f(\yB)\vert \le L\Vert \xB-\yB\Vert ^2\]
		\[ \therefore \vert \nabla \phi(\xB)-\nabla \phi(\yB)\vert=\vert \nabla f(\xB)-\nabla f(\yB)\vert \le L\Vert \xB-\yB\Vert ^2\]
		\[ \therefore \phi(y)\in C_L^{1,1}(\mathbb{R}^n)\]
	\end{enumerate}

		So $\phi(y)\in \mathcal{F}_L^{1,1}(\mathbb{R}^n)$


		We can easily find that $\nabla \phi(\bm{x_0}) = \nabla f(\bm{x_0})-\nabla f(\bm{x_0}) = 0$. 
		
		Since of its convexity, $y^*=\bm{x_0}$  is optimal point. 


\end{PROOF}

\begin{excercise}\label{e5}
Proof that, for $f: \RBB^n \rightarrow \RBB$ and $\alpha$ from $[0,1]$,  if
\begin{align*} 
	\alpha f(\xB) + (1-\alpha) f(\yB) &\geq f( \alpha \xB + (1-\alpha) \yB) \nonumber \\
	&+ \frac{\alpha(1-\alpha)}{2L} \normgen{\nabla f(\xB) - \nabla f(\yB)}^2, 
\end{align*}
then $f \in \functiontype{L}{1}{1}$.
\end{excercise}
\begin{PROOF}{e5}The answer is as follows:

	To prove that $f \in \functiontype{L}{1}{1}$, we need to prove that 
	$f$ is convex and $f \in C_{L}^{1,1}(\RBB^n)$

	\begin{enumerate}
		\item $f$ is convex
		
			\begin{align*} 
				\alpha f(\xB) + (1-\alpha) f(\yB) &\geq f( \alpha \xB + (1-\alpha) \yB) \nonumber + \frac{\alpha(1-\alpha)}{2L} \normgen{\nabla f(\xB) - \nabla f(\yB)}^2\\
				&\geq f( \alpha \xB + (1-\alpha) \yB) 
			\end{align*}
			So $f$ is convex.
		\item $f \in C_{L}^{1,1}(\RBB^n)$
			\begin{equation}
				\alpha f(\xB) + (1-\alpha) f(\yB) \geq f( \alpha \xB + (1-\alpha) \yB) \nonumber+ \frac{\alpha(1-\alpha)}{2L} \normgen{\nabla f(\xB) - \nabla f(\yB)}^2\\
			\end{equation}
			\begin{equation}
				f(\yB)\ge \frac{f(\alpha \xB + (1-\alpha) \yB)-\alpha f(\xB)}{1-\alpha}+\frac{\alpha}{2L} \Vert \nabla f(\xB) - \nabla f(\yB)\Vert ^2
			\end{equation}
			\begin{align}
				\therefore f(\yB) &= \lim_{\alpha\rightarrow 1}f(\yB)\\
				&\ge \lim_{\alpha\rightarrow 1} \left(\frac{f(\alpha \xB + (1-\alpha) \yB)-\alpha f(\xB)}{1-\alpha}+\frac{\alpha}{2L} \Vert \nabla f(\xB) - \nabla f(\yB)\Vert ^2\right)\\
				&\ge \lim_{\alpha\rightarrow 1} \left(\frac{f(\alpha \xB + (1-\alpha) \yB)-f(\xB) + (1-\alpha) f(\xB)}{1-\alpha}\right)+\frac{1}{2L} \Vert \nabla f(\xB) - \nabla f(\yB)\Vert ^2\\
				&\ge f(\xB)+\lim_{\alpha\rightarrow 1} \left(\frac{f(\alpha \xB + (1-\alpha) \yB)-f(\xB)}{1-\alpha}\right)+\frac{1}{2L}\Vert \nabla f(\xB)-\nabla f(\yB)\Vert ^2\\
				&\ge f(\xB)+\lim_{\alpha\rightarrow 1} \left(\frac{\innerproduct{\xB-\yB}{\nabla f(\alpha \xB + (1-\alpha) \yB)}}{-1}\right)+\frac{1}{2L}\Vert \nabla f(\xB)-\nabla f(\yB)\Vert ^2\\
				&\ge f(\xB)+\langle\nabla f(\xB),\yB-\xB\rangle+\frac{1}{2L}\Vert \nabla f(\xB)-\nabla f(\yB)\Vert ^2
			\end{align}
			swap $\xB,\yB$ we can get:
			\begin{equation}
				f(\xB)\ge f(\yB)+\langle\nabla f(\yB),\xB-\yB\rangle+\frac{1}{2L}\Vert \nabla f(\yB)-\nabla f(\xB)\Vert ^2
			\end{equation}
			And we plus above two equations, we can get:
			\begin{align}
				\frac{1}{L}\Vert \nabla f(\xB)-\nabla f(\yB)\Vert ^2&\le \langle\nabla f(\yB)-\nabla f(\xB),\yB-\xB\rangle\\
				&\le \Vert \nabla f(\yB)-\nabla f(\xB)\Vert \Vert \yB-\xB\Vert
			\end{align}

			So we get: 

			\[\vert \nabla f(\xB)-\nabla f(\yB)\vert \le L\Vert \yB-\xB \Vert\]
			So, $f\in C_{L}^{1,1}(\mathbb{R}^n)$
	\end{enumerate}
	So, $f\in \mathcal{F}_{L}^{1,1}(\mathbb{R}^n)$




\end{PROOF}

\begin{excercise}\label{e6}
	Proof that, for $f: \RBB^n \rightarrow \RBB$ and $\alpha$ from $[0,1]$,  if
\begin{align*} 
	0  \leq  \alpha f(\xB) + (1-\alpha) f(\yB)  &-  f( \alpha \xB + (1-\alpha) \yB) \nonumber \\
	&\leq \alpha (1-\alpha) \frac{L}{2} \normgen{\xB - \yB}^2,
\end{align*}
	then $f \in \functiontype{L}{\bm{1}}{\bm{1}}$.
\end{excercise}
\begin{PROOF}{e6}The answer is as follows:

	We make equivalent transformation for the inequality given above:
	\begin{equation}
		f(\yB)\le \frac{f(\alpha \xB+(1-\alpha)\yB)-\alpha f(\xB)}{1-\alpha}+\frac{\alpha L}{2}\Vert \yB-\xB\Vert ^2
	\end{equation}

	\begin{align}
		f(\yB) &= \lim_{\alpha \rightarrow 1}f(\yB)\\
		&\le \lim_{\alpha \rightarrow 1} \left(\frac{f(\alpha \xB + (1-\alpha) \yB)-\alpha f(\xB)}{1-\alpha}+\frac{\alpha L}{2}\Vert \yB-\xB\Vert ^2\right)\\
		&\le \lim_{\alpha \rightarrow 1} \left(\frac{f(\alpha \xB + (1-\alpha) \yB)-f(\xB)+(1-\alpha)f(\xB)}{1-\alpha}\right)+\frac{L}{2}\Vert \yB-\xB\Vert ^2\\
		&\le f(\xB)+\lim_{\alpha \rightarrow 1} \left(\frac{\innerproduct{\xB-\yB}{\nabla f(\alpha \xB + (1-\alpha) \yB)}}{-1}\right)+\frac{L}{2}\Vert \yB-\xB\Vert ^2\\
		&\le f(\xB)+\langle\nabla f(\xB),\yB-\xB\rangle+\frac{L}{2}\Vert \yB-\xB\Vert ^2
	\end{align}
	Simliarly,we can get:
	\begin{align}
		f(\yB) &= \lim_{\alpha \rightarrow 1}f(\yB)\\
		&\ge \lim_{\alpha \rightarrow 1} \frac{f(\alpha \xB + (1-\alpha) \yB)-\alpha f(\xB)}{1-\alpha}\\
		&\ge \lim_{\alpha \rightarrow 1} \left(\frac{f(\alpha \xB + (1-\alpha) \yB)-f(\xB)+(1-\alpha)f(\xB)}{1-\alpha}\right)\\
		&\ge f(\xB)+\lim_{\alpha \rightarrow 1} \left(\frac{\innerproduct{\yB-\xB}{\nabla f(\alpha \xB + (1-\alpha) \yB)}}{-1}\right)\\
		&\ge f(\xB)+\langle\nabla f(\yB),\xB-\yB\rangle
	\end{align}
	That is:
	\begin{equation}
		f(\xB)+\langle\nabla f(\yB),\xB-\yB\rangle \le f(\yB)\le f(\xB)+\langle\nabla f(\xB),\yB-\xB\rangle+\frac{L}{2}\Vert \yB-\xB\Vert ^2
	\end{equation}
	\begin{equation}
		0 \le f(\yB) - f(\xB) - \langle\nabla f(\yB)-\nabla f(\xB),\yB-\xB\rangle \le \frac{L}{2}\Vert \yB-\xB\Vert ^2
	\end{equation}
	Acctually, it follows from the definition of convex functions and Lemma (1.2.3) of Nesterov[2003].It is the equivalent of $f\in \mathcal{F}_L^{1,1}(\mathbb{R}^n)$. 
	
	So $f\in \mathcal{F}_L^{1,1}(\mathbb{R}^n)$.

\end{PROOF}

\end{document}