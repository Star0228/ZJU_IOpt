\documentclass{article}
\usepackage{hwopt}
\usepackage{amsmath}	% Package for AMS
\usepackage{amsthm}     % Package for AMS-therom
\usepackage{amssymb}	% Package for AMS-symbol
\usepackage{bm}

\newcommand{\aB}{\bm{a}}
\newcommand{\bB}{\bm{b}}
\newcommand{\xB}{\bm{x}}
\newcommand{\yB}{\bm{y}}
\newcommand{\vB}{\bm{v}}
\newcommand{\uB}{\bm{u}}
\newcommand{\gB}{\bm{g}}
\newcommand{\RBB}{\mathbb{R}}
\newcommand{\FM}{\mathcal{F}}
\newcommand{\SM}{\mathcal{S}}
\newcommand{\LM}{\mathcal{L}}
\newcommand{\domf}{\textrm{dom}\;f}
\newcommand{\functiontype}[3]{\FM_{#1}^{#2,#3}(\RBB^n)}
\newcommand{\normgen}[1]{\left\| #1 \right\|}
\newcommand{\normone}[1]{\left\| #1 \right\|_1}
\newcommand{\strongconvextype}[2]{\SM_{#1}^{#2}(\RBB^n)}
\newcommand{\argmin}{\mathop{\rm argmin}}

%%%%%%%%%%%%%%%%%
%     Title     %
%%%%%%%%%%%%%%%%%
\title{Coursework (5) for \emph{Introductory Lectures on Optimization}}
\author{Xiaoyu Wang \\ 3220104364}
\date{\today}

\begin{document}
\maketitle

\begin{excercise}\label{e0}
Prove the following theorem: \\
Let $\normgen{\cdot}$ be a vector norm in $\RBB^n$,  then 
\[
\partial \normgen{\cdot} = \left\{ V(\xB) \triangleq \left\{  \vB \in \RBB^n \big| \innerproduct{\vB}{\xB} = \normgen{\xB}, \normgen{\vB}_* \leq 1  \right\} \right\},
\]
where $\normgen{\vB}_*$ is the dual norm of $\normgen{\cdot}$, defined as
\[
\normgen{\vB}_* \triangleq \sup_{ \normgen{\uB} \leq 1} \innerproduct{\vB}{\uB}.
\]	
\end{excercise}
\begin{PROOF}{e0}The answer is as follows:

	We prove the theorem by the $V(\xB)\subset\partial \normgen{\xB}$ and $\partial \normgen{\xB}\subset V(\xB)$ 

	\begin{enumerate}
		\item $V(\xB)\subset\partial \normgen{\xB}$
		
		Let $\vB \in V(\xB)$, and $\forall \yB \in \RBB^n$, we have

		\begin{align}
			\normgen{\xB} + \innerproduct{\vB}{\yB-\xB} &= \normgen{\xB} + \innerproduct{\vB}{\yB} - \innerproduct{\vB}{\xB} \\
			&= \normgen{\xB} + \innerproduct{\vB}{\yB} - \normgen{\xB} \\
			&= \innerproduct{\vB}{\yB} \\
			&\leq \normgen{\yB} \cdot \normgen{\vB}_*  \\
			&\leq \normgen{\yB}
		\end{align}

		(2)(5)'s reason is that $\vB \in V(\xB)$.

		(4)'s reason is that $\textbf{Holder's inequality}$.

		That is $\normgen{\yB}\geq \normgen{\xB} + \innerproduct{\vB}{\yB-\xB}$,
		since of arbitrariness of $\yB$, it is the definition of sub-gradient of 
		$\normgen{\cdot}$ and $\vB \in \partial \normgen{\xB}$.

		So $V(\xB)\subset\partial \normgen{\xB}$.
		\item $\partial \normgen{\xB}\subset V(\xB)$
		
		Let $\vB \in \partial \normgen{\xB}$, and $\forall \yB \in \RBB^n$, we have:
		\begin{equation}
			\normgen{\yB} \geq \normgen{\xB} + \innerproduct{\vB}{\yB-\xB}
		\end{equation}
		\begin{equation}
			\therefore \forall \yB \in \RBB^n, \innerproduct{\vB}{\yB} - \normgen{\yB} \leq \innerproduct{\vB}{\xB} - \normgen{\xB}
		\end{equation}
		\begin{equation}
			\therefore \normgen{\vB}_* \text{Fenchel conjugate} \triangleq  \sup_{\yB} \left\{  \innerproduct{\vB}{\yB} -\normgen{\yB}\right\} \leq \innerproduct{\vB}{\xB} - \normgen{\xB}
		\end{equation}
		So we get:
		\begin{equation}
			\normgen{\vB}_* = \begin{cases}
				0&\normgen{\vB}_* \leq 1\\
				+\infty&\normgen{\vB}_* > 1
			\end{cases}
		\end{equation}
		Since $\innerproduct{\vB}{\xB}-\normgen{\xB}$ is always finite, we have $\normgen{\vB}_* \leq 1$.
		\begin{align}
			0 &\leq \innerproduct{\vB}{\xB} - \normgen{\xB} \\
			&\leq \normgen{\xB} \cdot \normgen{\vB}_* -\normgen{\xB}\\
			&\leq \normgen{\xB} \cdot 1 -\normgen{\xB}\  \because \normgen{\vB}_* \leq 1\\
			&= 0
		\end{align}
		So we have $\innerproduct{\vB}{\xB} = \normgen{\xB}$,that implies $\partial \normgen{\xB}\subset V(\xB)$.
	\end{enumerate}
	Therefore, we have $\partial \normgen{\xB} = V(\xB)$.

\end{PROOF}

\begin{excercise}\label{e1}
Write down the subdifferentials of following functions. 
\begin{enumerate}
	\item $f(\xB) = |\xB|, \xB \in \RBB^1$.\vspace{8pt}
	\item $f(\xB) = \sum_{i=1}^{m} | \innerproduct{\aB_i}{\xB} - \bB_i|$.\vspace{8pt}
	\item $f(\xB) = \max_{1 \leq i \leq n} \xB^{(i)}$.\vspace{8pt}
	\item $f(\xB) = \normgen{\xB}$.\vspace{8pt}
	\item $f(\xB) = \normone{\xB} = \sum_{i=1}^{n} |\xB^{(i)}|$.
\end{enumerate}
\end{excercise}
\begin{SOLUTION}{e1}The answer is as follows:

	\begin{enumerate}
		\item $f(\xB) = |\xB|, \xB \in \RBB^1$.
		
		\begin{equation}
			f(\xB) = \max_{-1\leq t \leq 1} \gB \cdot \xB \Rightarrow \partial f(\bm{0}) = [-1, 1]
		\end{equation}

		\item $f(\xB) = \sum_{i=1}^{m} | \innerproduct{\aB_i}{\xB} - \bB_i|$
		
			Denote
			\[
				\begin{aligned}
					I_-(\xB) &= \{ i \mid \innerproduct{\aB_i}{\xB_i} - \bB_i < 0 \},\\
					I_+(\xB) &= \{ i \mid \innerproduct{\aB_i}{\xB_i} - \bB_i > 0 \},\\
					I_0(\xB) &= \{ i \mid \innerproduct{\aB_i}{\xB_i} - \bB_i = 0 \}.
				\end{aligned}	
			\]
			Then 
			\[
				\partial f(\xB)	= \sum_{i \in I_+(\xB)} \bm{a}_i - \sum_{i \in I_-(\xB)} \bm{a}_i + \sum_{i \in I_0(\xB)} [-\bm{a}_i, \bm{a}_i].
			\]
		\item $f(\xB) = \max_{1 \leq i \leq n} \xB^{(i)}$
		
			Denote:$I(\xB) = \{ i \mid \xB^{(i)} = f(\xB) \}$
			
			Then we have:
			\[
				\partial f(\xB) = \begin{cases}
					\mathrm{Conv} \{ \bm{e}_i \mid 1 \le i \le n \} \equiv \Delta_n, & \xB = \bm{0},\\
					\mathrm{Conv} \{ \bm{e}_i \mid i \in I(\xB) \}, & \xB \ne \bm{0}.	
				\end{cases}
			\]
		\item $f(\xB) = \normgen{\xB}$
		
			\[
				\partial f(\xB)	= \begin{cases}
					B_2(0, 1) = \{ \xB \in \RBB^n \mid \|\xB\| \le 1 \}, &\xB = 0,\\
					\{\xB / \|\xB\|\}, &\xB \ne 0.
				\end{cases}
			\]
		\item $f(\xB) = \normone{\xB} = \sum_{i=1}^{n} |\xB^{(i)}|$
			
		Denote:
			\[
				\begin{aligned}
					I_+(\xB) &= \{ i \mid \xB^{(i)} > 0 \},\\
					I_-(\xB) &= \{ i \mid \xB^{(i)} < 0 \},\\
					I_0(\xB) &= \{ i \mid \xB^{(i)} = 0 \}.
				\end{aligned}	
			\]
			Then we have
			\[
				\partial f(\xB)	= \begin{cases}
					B_\infty(0, 1) = \{ \xB \in \RBB^n \mid \max_{1 \le i \le n} |\xB^{(i)}| \le 1 \}, &\xB = 0,\\
					\sum_{i \in I_+(\xB)} \bm{e}_i - \sum_{i \in I_-(\xB)} \bm{e}_i + \sum_{i \in I_0(\xB)} [-\bm{e}_i, \bm{e}_i], &\xB \ne 0.
				\end{cases}
			\]
	\end{enumerate}

\end{SOLUTION}

\begin{excercise}\label{e2}
Please write down three sequences and prove that they satisfy the following conditions:
\[
 h_k > 0, h_k \to 0, \sum_{k=0}^{\infty} h_k = \infty.
 \]
\end{excercise}

\begin{SOLUTION}{e2}The answer is as follows:
	
	\begin{enumerate}
		\item $h_k = \frac{1}{k+1}$

			\begin{align}
				h_k &= \frac{1}{k+1} >0 \\ 
				\lim_{k \to \infty} h_k &= \lim_{k \to \infty}\frac{1}{k+1} = 0\\
				\sum_{k=0}^{\infty} h_k &= \sum_{k=0}^{\infty} \frac{1}{k+1} = \infty.
				\because \sum_{k=0}^{\infty} \frac{1}{k+1} \text{is }\textbf{harmonic progression}.
			\end{align}
				
		\item $h_k = \frac{1}{\sqrt{k+1} }$

			\begin{align}
				h_k &= \frac{1}{\sqrt{k+1}} >0 \\ 
				\lim_{k \to \infty} h_k &= \lim_{k \to \infty}\frac{1}{\sqrt{k+1}} = 0\\
				\sum_{k=0}^{\infty} h_k &= \sum_{k=0}^{\infty} \frac{1}{\sqrt{k+1}} \ge \sum_{k=0}^{\infty} \frac{1}{k+1} = \infty.
			\end{align}
		\item $h_k = \frac{1}{\ln{(k+2)} }$

			\begin{align}
				h_k &= \frac{1}{\ln{(k+2)}} >0 \\ 
				\lim_{k \to \infty} h_k &= \lim_{k \to \infty}\frac{1}{\ln{(k+2)}} = 0\\
				\sum_{k=0}^{\infty} h_k &= \sum_{k=0}^{\infty} \frac{1}{\ln{(k+2)}} > \sum_{k=0}^{\infty} \frac{1}{k+1} = \infty.
			\end{align}
	\end{enumerate}
\end{SOLUTION}
\end{document}